% !TeX program = xelatex
\documentclass[english,10pt,openany,a5paper]{book}
\usepackage{babel}
\usepackage{retrotype}
\usepackage{adjustbox}
\usepackage{layouts}
\usepackage{subfig}
\usepackage{afterpage}
 \usepackage[toc]{multitoc}
\usepackage{graphicx}
\usepackage{multicol}
\usepackage{afterpage}
\usepackage{makecell}
\usepackage{float}

%%%%%%%%%%%%%%%%%%%%%%%%%%%
\usepackage{fancyhdr} % Custom headers and footers
\fancypagestyle{fancyplain}{%
	\fancyhf{}% clear all header and footer fields
	\fancyfoot[C]{\thepage} % except the center
	\fancyhead[C]{\leftmark}
}

\pagestyle{fancyplain}
%%%%%%%%%%%%%%%%%%%%%%%%%%%

\usepackage{pgffor, ifthen}
\newcommand{\notes}[3][\empty]{%
	\noindent Notes\vspace{10pt}\\
	\foreach \n in {1,...,#2}{%
		\ifthenelse{\equal{#1}{\empty}}
		{\rule{#3}{0.5pt}\\}
		{\rule{#3}{0.5pt}\vspace{#1}\\}
	}
}

\newcommand{\tablex}{0.985}
\newcommand{\tablesm}{0.47}


\newcommand\blankpage{%
	\newpage 
	
	\thispagestyle{empty}%
	\
	
	\newpage }

\newcommand\tripip{\textbf{\op{\_}{/\textbackslash}} }

\newcommand\cirpip{\textbf{\op{O}{@}} }


% HYPERSETUP

\usepackage{hyperref}
\usepackage{bookmark}

\def\title{Rosethorn Keep}
\def\author{Charlotte Laskowski and Kaydevs}
\def\rtVer{1.0}
\def\rtDate{2024-06-15}

\hypersetup{%
	pdfborderstyle={/S/U/W 0},% underlined hyperlinks
	bookmarksdepth=5,% paragraph level depth bookmarks
	pdftitle={\title},
	pdfauthor={\author},
	%pdfkeywords={\keywords},
}

\smalltitles

\begin{document}
	
		\begin{titlepage}
	\begin{figure}%
		\centering
		\subfloat{\BVerbatimInput{img/rosethorn.txt}}%
	\end{figure}
	\begin{figure}%
		\centering
		\subfloat{{\BVerbatimInput{img/castle.txt} }}%
		\quad
		\subfloat{{\BVerbatimInput{img/keep.txt} }}%
	\end{figure}
	
	\begin{adjustbox}{center}
		\linechars[-]{7}
		{\hfill{}by \author{}}
		\linechars[-]{7}
	\end{adjustbox}
\end{titlepage}
	
	\blankpage
	
	\section{Credits}
	
	\thispagestyle{empty}
	
	Writing, layout: Charlotte Laskowski\\
	Writing: Kaydevs\\
	Editing: Jarrett Crader
	
	\skipline\skipline
	\skipline
	
	\href{https://selkie.itch.io/rosethorn-keep}{selkie.itch.io/rosethorn-keep}\\
	\href{https://dreamware.press}{dreamware.press}
	
	\newpage
	

	
	\toc[2]
	
	\thispagestyle{empty}
	
	
	\chapter{Tutorial}
	
	\thispagestyle{empty}
	
	
	\header{Materials}
	\tld  A sheet of paper and pencil, or something else to draw on.
	
	\tld  Some dice. A D6 will be required for the game. Online dice rollers are perfectly acceptable.
	
	\skipline
	\begin{adjustbox}{left}
		\em Reclaim what’s rightfully yours...\em
	\end{adjustbox}
	\begin{adjustbox}{right}
		\em Reclaim Rosethorn Keep...! \em
	\end{adjustbox}
	
	\input{sections/tutorial-a5.tex}
	
	\chapter{The Castle}

\thispagestyle{empty}

\begin{tabbox}[4][4]
	\subsubsection{Great Hall}
	\tabpar[3]{Determines the maximum tier you can upgrade other sections to.}
	\tabpar[3]{\textbf{Resources:} Gold \& Stone}
	
	\subsubsection{Armory}
	\tabpar[3]{-1 per tier from the TN of \tripip Combat events.}
	\tabpar[3]{\textbf{Resources:} Iron \& Lumber}
	
	\subsubsection{Courtyard}
	\tabpar[3]{-1 per tier from the TN of \cirpip Noncombat events.}
	\tabpar[3]{\textbf{Resources:} Stone \& Iron}
	
	\subsubsection{Kitchen}
	\tabpar[3]{Choose a number of dice equal to tier to reroll at dungeon start.}
	\tabpar[3]{\textbf{Resources:} Stone \& Lumber}
	
	\subsubsection{Stables}
	\tabpar[3]{Bring back a number of Resources from the dungeon equal to tier.}
	\tabpar[3]{\textbf{Resources:} Lumber \& Leather}
	
	\subsubsection{Bedrooms}
	\tabpar[3]{Regenerate more companions for the next dungeon, equal to tier + 1.}
	\tabpar[3]{\textbf{Resources:} Leather \& Gold}
\end{tabbox}

\pagebreak

\subsection{Great Hall}

Determines the maximum tier you can upgrade other sections to.
\begin{tabbox}[4][4]
	
\textbf{Tier 0}\\

	\tabpar[3]{N/A}
	\tabpar[3]{\textbf{Upgrade Cost:} N/A}\skipline

\textbf{Tier 1}\\

	\tabpar[3]{Buildings may be up to Tier 1.}
	\tabpar[3]{\textbf{Upgrade Cost:} N/A}\skipline
	
\textbf{Tier 2}\\

	\tabpar[3]{Buildings may be up to Tier 2.}
	\tabpar[3]{Roll 1d6 to determine your next companion.}
	\tabpar[3]{\textbf{Upgrade Cost:} 10 Gold \& 5 Stone}\skipline
	
\textbf{Tier 3}\\

	\tabpar[3]{Buildings may be up to Tier 3.}
	\tabpar[3]{Roll 2d6 to determine your next companion.}
	\tabpar[3]{\textbf{Upgrade Cost:} 15 Gold \& 10 Stone}\skipline
	
\textbf{Tier 4}\\

	\tabpar[3]{Buildings may be up to Tier 4.}
	\tabpar[3]{Roll 3d6 to determine your next companion.}
	\tabpar[3]{\textbf{Upgrade Cost:} 20 Gold \& 15 Stone}\skipline
	
\textbf{Tier 5}\\

	\tabpar[3]{Buildings may be up to Tier 5.}
	\tabpar[3]{Roll 3d6 and drop the lowest result to determine your next companion.}
	\tabpar[3]{\textbf{Upgrade Cost:} 30 Gold \& 30 Stone}\skipline
\end{tabbox}

\pagebreak


\subsection{Armory}

-1 per tier from the TN of \tripip Combat events.
\skipline

Example: You roll a 4 for the event while exploring the Crypt Dungeon, and come across a skeleton. The TN for this encounter is 4, plus the tier of your Great Hall. If your Great Hall is tier 2, the TN of the encounter. When including the Armory’s benefit, the final TN of that encounter is 4.

\begin{tabbox}[4][4]
	
	\textbf{Tier 0}\\
	
	\tabpar[3]{N/A}
	\tabpar[3]{\textbf{Upgrade Cost:} N/A}\skipline
	
	\textbf{Tier 1}\\
	
	\tabpar[3]{-1 from the TN of \tripip Combat events.}
	\tabpar[3]{\textbf{Upgrade Cost:} 5 Iron \& 1 Lumber}\skipline
	
	\textbf{Tier 2}\\
	
	\tabpar[3]{-1 from the TN of \tripip Combat events.}
	\tabpar[3]{\textbf{Upgrade Cost:} 10 Iron \& 5 Lumber}\skipline
	
	\textbf{Tier 3}\\
	
	\tabpar[3]{-1 from the TN of \tripip Combat events.}
	\tabpar[3]{\textbf{Upgrade Cost:} 15 Iron \& 10 Lumber}\skipline
	
	\textbf{Tier 4}\\
	
	\tabpar[3]{-1 from the TN of \tripip Combat events.}
	\tabpar[3]{\textbf{Upgrade Cost:} 20 Iron \& 15 Lumber}\skipline
	
	\textbf{Tier 5}\\
	
	\tabpar[3]{-1 from the TN of \tripip Combat events.}
	\tabpar[3]{\textbf{Upgrade Cost:} 30 Iron \& 30 Lumber}\skipline
\end{tabbox}

\pagebreak


\subsection{Courtyard}

-1 per tier from the TN of \cirpip Noncombat events.
\skipline

Example: You roll a 4 for the event while exploring the Monster Lair, and come across a hunting dog. The TN for this encounter is 4, plus the tier of your Great Hall. If your Great Hall is tier 2, the TN of the encounter. When including the Courtyard’s benefit, the final TN of that encounter is 4.

\skipline

\begin{tabbox}[4][4]
	
	\textbf{Tier 0}\\
	
	\tabpar[3]{N/A}
	\tabpar[3]{\textbf{Upgrade Cost:} N/A}\skipline
	
	\textbf{Tier 1}\\
	
	\tabpar[3]{-1 from the TN of \cirpip Noncombat events.}
	\tabpar[3]{\textbf{Upgrade Cost:} 5 Stone \& 1 Iron}\skipline
	
	\textbf{Tier 2}\\
	
	\tabpar[3]{-1 from the TN of \cirpip Noncombat events.}
	\tabpar[3]{\textbf{Upgrade Cost:} 10 Stone \& 5 Iron}\skipline
	
	\textbf{Tier 3}\\
	
	\tabpar[3]{-1 from the TN of \cirpip Noncombat events.}
	\tabpar[3]{\textbf{Upgrade Cost:} 15 Stone \& 10 Iron}\skipline
	
	\textbf{Tier 4}\\
	
	\tabpar[3]{-1 from the TN of \cirpip Noncombat events.}
	\tabpar[3]{\textbf{Upgrade Cost:} 20 Stone \& 15 Iron}\skipline
	
	\textbf{Tier 5}\\
	
	\tabpar[3]{-1 from the TN of \cirpip Noncombat events.}
	\tabpar[3]{\textbf{Upgrade Cost:} 30 Stone \& 30 Iron}\skipline
\end{tabbox}

\pagebreak


\subsection{Kitchen}

Choose a number of recorded dice to reroll at dungeon start.
\skipline

\begin{tabbox}[4][4]
	
	\textbf{Tier 0}\\
	
	\tabpar[3]{N/A}
	\tabpar[3]{\textbf{Upgrade Cost:} N/A}\skipline
	
	\textbf{Tier 1}\\
	
	\tabpar[3]{Reroll up to 1 die at the beginning of a dungeon, only once per die.}
	\tabpar[3]{\textbf{Upgrade Cost:} 5 Stone \& 1 Lumber}\skipline
	
	\textbf{Tier 2}\\
	
	\tabpar[3]{Reroll up to 2 dice at the beginning of a dungeon, only once per die.}
	\tabpar[3]{\textbf{Upgrade Cost:} 10 Stone \& 5 Lumber}\skipline
	
	\textbf{Tier 3}\\
	
	\tabpar[3]{Reroll up to 3 dice at the beginning of a dungeon, only once per die.}
	\tabpar[3]{\textbf{Upgrade Cost:} 15 Stone \& 10 Lumber}\skipline
	
	\textbf{Tier 4}\\
	
	\tabpar[3]{Reroll up to 4 dice at the beginning of a dungeon, only once per die.}
	\tabpar[3]{\textbf{Upgrade Cost:} 20 Stone \& 15 Lumber}\skipline
	
	\textbf{Tier 5}\\
	
	\tabpar[3]{Reroll up to 5 dice at the beginning of a dungeon, only once per die.}
	\tabpar[3]{\textbf{Upgrade Cost:} 30 Stone \& 30 Lumber}\skipline
\end{tabbox}

\pagebreak


\subsection{Stables}

Bring back extra Resources.
\skipline

\begin{tabbox}[4][4]
	
	\textbf{Tier 0}\\
	
	\tabpar[3]{N/A}
	\tabpar[3]{\textbf{Upgrade Cost:} N/A}\skipline
	
	\textbf{Tier 1}\\
	
	\tabpar[3]{Bring back 1 more of each resource gained.}
	\tabpar[3]{\textbf{Upgrade Cost:} 5 Lumber \& 1 Leather}\skipline
	
	\textbf{Tier 2}\\
	
	\tabpar[3]{Bring back 2 more of each resource gained.}
	\tabpar[3]{\textbf{Upgrade Cost:} 10 Lumber \& 5 Leather}\skipline
	
	\textbf{Tier 3}\\
	
	\tabpar[3]{Bring back 3 more of each resource gained.}
	\tabpar[3]{\textbf{Upgrade Cost:} 15 Lumber \& 10 Leather}\skipline
	
	\textbf{Tier 4}\\
	
	\tabpar[3]{Bring back 4 more of each resource gained.}
	\tabpar[3]{\textbf{Upgrade Cost:} 20 Lumber \& 15 Leather}\skipline
	
	\textbf{Tier 5}\\
	
	\tabpar[3]{Bring back 5 more of each resource gained.}
	\tabpar[3]{\textbf{Upgrade Cost:} 30 Lumber \& 30 Leather}\skipline
\end{tabbox}

\pagebreak


\subsection{Bedrooms}

Regenerate more companions between dungeons after they are defeated.
\skipline

\begin{tabbox}[4][4]
	
	\textbf{Tier 0}\\
	
	\tabpar[3]{Regenerate 1 companion between dungeons.}
	\tabpar[3]{\textbf{Upgrade Cost:} N/A}\skipline
	
	\textbf{Tier 1}\\
	
	\tabpar[3]{Regenerate 2 companions between dungeons.}
	\tabpar[3]{\textbf{Upgrade Cost:} 5 Leather \& 1 Gold}\skipline
	
	\textbf{Tier 2}\\
	
	\tabpar[3]{Regenerate 3 companions between dungeons.}
	\tabpar[3]{\textbf{Upgrade Cost:} 10 Leather \& 5 Gold}\skipline
	
	\textbf{Tier 3}\\
	
	\tabpar[3]{Regenerate 4 companions between dungeons.}
	\tabpar[3]{\textbf{Upgrade Cost:} 15 Leather \& 10 Gold}\skipline
	
	\textbf{Tier 4}\\
	
	\tabpar[3]{Regenerate 5 companions between dungeons.}
	\tabpar[3]{\textbf{Upgrade Cost:} 20 Leather \& 15 Gold}\skipline
	
	\textbf{Tier 5}\\
	
	\tabpar[3]{Regenerate 6 companions between dungeons.}
	\tabpar[3]{\textbf{Upgrade Cost:} 30 Leather \& 30 Gold}\skipline
\end{tabbox}

\pagebreak
	
	
	\chapter{Dungeons}

\thispagestyle{empty}

To calculate the total Resources you bring back to the castle, multiply the dungeon’s Base Resources by the number of events completed successfully. Remember to add the tier of your Stables to both Resources before multiplying by the amount of successful events.

\skipline

At Tiers 2 \& 3, roll \textbf{2d6} per companion to determine the event. At Tiers 4 \& 5, roll \textbf{3d6} per companion to determine the event.

\skipline

If you bring more than five companions, you also fight the Boss. This is automatic and replaces one of the dungeon’s events. To defeat bosses, companions must overcome both TNs listed.

\begin{multicols}{2}

	\subsubsection{Abandoned? Mine}
	\tabpar[1]{Though taken over by a beast of sulfur, the materials needed for your forge lie within.}
	\tabpar[1]{\textbf{Resources:} 5 Iron \& 3 Stone}
	
	\subsubsection{Crypt}
	\tabpar[1]{An old tomb, now overrun by undead. Within, many treasures await.}
	\tabpar[1]{\textbf{Resources:} 5 Gold \& 3 Iron}
	
	\columnbreak
	
	\subsubsection{Haunted Forest}
	\tabpar[1]{Cursed by an ancient evil, and yet still home to bandits.}
	\tabpar[1]{\textbf{Resources:} 5 Lumber \& 3 Leather}
	
	\subsubsection{Monster Lair}
	\tabpar[1]{The lair of a great white beast. Holds the treasures of misfortune seekers.}
	\tabpar[1]{\textbf{Resources:} 5 Leather \& 3 Gold}
	
\end{multicols}
	
	\subsubsection{Ruins}
	\tabpar[1]{There are many abandoned towns near the castle. Old secrets hide here.}
	\tabpar[1]{\textbf{Resources:} 5 Stone \& 3 Lumber}


\pagebreak

\input{sections/dungeon-mine-a5.tex}

\pagebreak

\input{sections/dungeon-crypt-a5.tex}

\pagebreak

\input{sections/dungeon-forest-a5.tex}

\pagebreak

\subsection{Monster Lair}

\begin{multicols}{2}
		
		\textbf{1. Hoard}\\
		
		\tabpar[1]{Treasure! You might find some neat stuff inside if its not enchanted.}
		\tabpar[1]{\textbf{TN:} MAG 3 + [TIER] \cirpip}\skipline
		
		\textbf{2. Fresh Tracks}\\
		
		\tabpar[1]{Follow the tracks if you are fast enough to catch who made them.}
		\tabpar[1]{\textbf{TN:} SPD 4 + [TIER] \cirpip}\skipline
		
		\textbf{3. Pile of Bones}\\
		
		\tabpar[1]{Search the bones for remaining valuables if you can stomach it.}
		\tabpar[1]{\textbf{TN:} DEF 4 + [TIER] \cirpip}\skipline
		
		\textbf{4. Lost Hound}\\
		
		\tabpar[1]{A lost hunting dog. You think its owner came monster hunting.}
		\tabpar[1]{\textbf{TN:} SPD 4 + [TIER] \cirpip}\skipline
		
		\textbf{5. Giant Rat}\\
		
		\tabpar[1]{He makes all of the rules.}
		\tabpar[1]{\textbf{TN:} ATK 5 + [TIER] \tripip}\skipline
		
		\textbf{6. Ambush}\\
		
		\tabpar[1]{Ambushed by an unknown foe, fight back the best you can.}
		\tabpar[1]{\textbf{TN:} ATK 6 + [TIER] \tripip}\skipline
		
		\columnbreak
		
		\textbf{7. Bear Trap}\\
		
		\tabpar[1]{Many bear traps lie between you and your destination.}
		\tabpar[1]{\textbf{TN:} SPD 5 + [TIER] \cirpip}\skipline
		
		\textbf{8. Signs}\\
		
		\tabpar[1]{Marked with warnings, interpret them to learn what’s ahead.}
		\tabpar[1]{\textbf{TN:} MAG 5 + [TIER] \cirpip}\skipline
		
		\textbf{9. Cave-in}\\
		
		\tabpar[1]{The lair’s ceiling is caving in. Get out before you’re crushed.}
		\tabpar[1]{\textbf{TN:} SPD 5 + [TIER] \cirpip}\skipline
		
		\textbf{10. Knightslayer Snail}\\
		
		\tabpar[1]{The mortal enemy of Knights and Squires: the snail.}
		\tabpar[1]{\textbf{TN:} DEF 5 + [TIER] \tripip}\skipline
		
		\textbf{11. Summoning Circle}\\
		
		\tabpar[1]{An incomplete summoning circle. You may be able to finish the ritual...}
		\tabpar[1]{\textbf{TN:} MAG 6 + [TIER] \cirpip}\skipline
		
		\textbf{12. Monster Hunter}\\
		
		\tabpar[1]{A hunter missing its hound. Have you seen it?}
		\tabpar[1]{\textbf{TN:} CHA 7 + [TIER] \cirpip}\skipline
		
		\columnbreak
		
		\textbf{13. Harpies}\\
		
		\tabpar[1]{They made a nest here, and you seem like easy prey.}
		\tabpar[1]{\textbf{TN:} ATK 6 + [TIER] \tripip}\skipline
		
		\textbf{14. Stalking Beast}\\
		
		\tabpar[1]{An unknown monster is looking for you, and you can’t fight it. Hide!}
		\tabpar[1]{\textbf{TN:} SPD 6 + [TIER] \tripip}\skipline
		
		\textbf{15. Cockatrice}\\
		
		\tabpar[1]{Be mindful of its petrifying gaze. You must counter its magic.}
		\tabpar[1]{\textbf{TN:} MAG 6 + [TIER] \tripip}\skipline
		
		\columnbreak
		
		\textbf{16. Corpse Pit}\\
		
		\tabpar[1]{Remains of meals from the boss, but you see the glint of gold...}
		\tabpar[1]{\textbf{TN:} DEF 6 + [TIER] \cirpip}\skipline
		
		\textbf{17. Loot Pile}\\
		
		\tabpar[1]{A glinting pile of gold. Try to take some, but be quick.}
		\tabpar[1]{\textbf{TN:} SPD 7 + [TIER] \cirpip}\skipline
		
		\textbf{18. Ogre}\\
		
		\tabpar[1]{A great man armed with a heavy club and an insatiable hunger.}
		\tabpar[1]{\textbf{TN:} ATK 8 + [TIER] \cirpip}\skipline
\end{multicols}
\textbf{BOSS: Rabbit} -- An unassuming but lethal beast that cannot be slayed by you or any.\\
\textbf{TN:} DEF 11 \tripip \& SPD 12 \cirpip

\pagebreak

\subsection{Ruins}

\begin{multicols}{2}
		
		\textbf{1. Old Cellar}\\
		
		\tabpar[1]{May contain treasures left behind, or rotting food.}
		\tabpar[1]{\textbf{TN:} DEF 3 + [TIER] \cirpip}\skipline
		
		\textbf{2. Hermit}\\
		
		\tabpar[1]{He’s lived alone for decades. Who knows what secrets he holds...}
		\tabpar[1]{\textbf{TN:} CHA 3 + [TIER] \cirpip}\skipline
		
		\textbf{3. Pack of Stray Dogs}\\
		
		\tabpar[1]{Left behind by their owners, now you’re in their territory.}
		\tabpar[1]{\textbf{TN:} SPD 4 + [TIER] \tripip}\skipline
		
		\textbf{4. Crumbling Room}\\
		
		\tabpar[1]{An old bedroom that falls apart when you enter.}
		\tabpar[1]{\textbf{TN:} SPD 4 + [TIER] \cirpip}\skipline
		
		\textbf{5. Bookshelf}\\
		
		\tabpar[1]{You can’t tell if it contains old grimoires or romance novels.}
		\tabpar[1]{\textbf{TN:} MAG 5 + [TIER] \cirpip}\skipline
		
		\textbf{6. Bandit}\\
		
		\tabpar[1]{A bandit looking to score some cash. You have some cash.}
		\tabpar[1]{\textbf{TN:} ATK 6 + [TIER] \tripip}\skipline
		
		\columnbreak
		
		\textbf{7. Tainted Well}\\
		
		\tabpar[1]{It sickens you when you take a sip. Resist the blight.}
		\tabpar[1]{\textbf{TN:} DEF 5 + [TIER] \cirpip}\skipline
		
		\textbf{8. Broken Windmill}\\
		
		\tabpar[1]{Someone must have confused it for a giant.}
		\tabpar[1]{\textbf{TN:} MAG 5 + [TIER] \cirpip}\skipline
		
		\textbf{9. Mimic}\\
		
		\tabpar[1]{Roll 1d6. If you have an equivalent die, steal its treasure.}\skipline
		
		\textbf{10. Dead Horse}\\
		
		\tabpar[1]{Maybe if you beat it enough, something interesting will happen.}
		\tabpar[1]{\textbf{TN:} ATK 5 + [TIER] \cirpip}\skipline
		
		\textbf{11. Hidden Room}\\
		
		\tabpar[1]{Equal chance of accessing it as there is of activating a trap.}
		\tabpar[1]{\textbf{TN:} SPD 6 + [TIER] \cirpip}\skipline
		
		\textbf{12. Bridge Troll}\\
		
		\tabpar[1]{Trying to extort you. Luckily, all he wants is coin.}
		\tabpar[1]{\textbf{TN:} ATK 7 + [TIER] \tripip}\skipline
		
		\columnbreak
		
		\textbf{13. Doppelganger}\\
		
		\tabpar[1]{One among you has been replaced with a doppelganger. Find it!}
		\tabpar[1]{\textbf{TN:} MAG 6 + [TIER] \tripip}\skipline
		
		\textbf{14. Zombies}\\
		
		\tabpar[1]{Living dead remains of the old villagers who didn’t leave.}
		\tabpar[1]{\textbf{TN:} ATK 6 + [TIER] \tripip}\skipline
		
		\textbf{15. Broken Wagon}\\
		
		\tabpar[1]{Someone must have beaten its horses.}
		\tabpar[1]{\textbf{TN:} DEF 6 + [TIER] \tripip}\skipline
		
		\columnbreak
		
		\textbf{16. Booby Trap}\\
		
		\tabpar[1]{Seems pretty hard to miss in hindsight, but you fell on it.}
		\tabpar[1]{\textbf{TN:} SPD 6 + [TIER] \cirpip}\skipline
		
		\textbf{17. Plague Pit}\\
		
		\tabpar[1]{The resting place of the ruin’s original inhabitants.}
		\tabpar[1]{\textbf{TN:} DEF 7 + [TIER] \cirpip}\skipline
		
		\textbf{18. Derelict Church}\\
		
		\tabpar[1]{An abandoned church. Something feels... wrong.}
		\tabpar[1]{\textbf{TN:} MAG 8 + [TIER] \tripip}
	
\end{multicols}

\textbf{BOSS: Ancient Robot} -- A machine made by the original villagers. Try to deactivate it.\\
\textbf{TN:} MAG 11 \cirpip\& DEF 12 \tripip

\pagebreak
	
	\input{sections/demonlord-a5.tex}
	
	\pagebreak
	
	\uheader{Map of Rosethorn Keep}
	
	\pagestyle{empty}
	
	\begin{adjustbox}{fbox=1pt}
		\begin{tikzpicture}
			\draw[step=1cm,color=black,dashed] (-1,-1) grid (10,5);
		\end{tikzpicture}
	\end{adjustbox}
	
	\uheader{Companions}
	\begin{retrotable}[|l|L|l|c|c|c|c|c|c|][\tablex\linewidth]
		\multicolumn{9}{_l_}{\linecharw[\_]{\tablex\linewidth}}\\
		\# & Companion & Stars & ATK & DEF & MAG  & CHA & SPD & Party \\ \hline
		1  & Archer        & *  & 2 & 1 & 2 & 2 & 3& ~   \\\hline
		2  & Arriaga the Rogue         & * & 3 & 2 & 1 & 2 & 2 & ~ \\\hline
		3  & Mathias the Squire      & *  & 2 & 3 & 2 & 2 & 1 & ~\\\hline
		4  & Chime the Bard          & *  & 1 & 2 & 2 & 3 & 2 & ~  \\\hline
		5  & Ellen the Warrior          & *  & 2 & 3 & 1 & 2 & 2 & ~  \\\hline
		6  & Lichen the Sling     & * & 3 & 1 & 2 & 2 & 2 & ~  \\\hline
		7  & Cassandra the Witch       & **  & 1 & 1 & 3 & 3 & 2& ~ \\\hline
		8  & Diego the Duelist       & **  & 3 & 1 & 1 & 2 & 3 & ~ \\\hline
		9  & Astrid, Shield Maiden  & ** & 3 & 3 & 1 & 1 & 2& ~ \\\hline
		10 & Eryn the Nun        & **     & 1 & 2 & 3 & 3 & 1 & ~ \\\hline
		11 & Scarlet, Plague Doc  & *** & 1 & 3 & 4 & 1 & 1& ~ \\\hline
		12 & Gottfried the Knight     & ***      & 3 & 4 & 1 & 1 & 1  & ~\\ \hline
		~ &~ &~ & ~& ~ & ~& ~& ~& ~\\ \hline
		~ &~ &~ & ~& ~ & ~& ~& ~& ~\\ \hline
		~ &~ &~ & ~& ~ & ~& ~& ~& ~\\ \hline
		~ &~ &~ & ~& ~ & ~& ~& ~& ~\\ \hline
		~ &~ &~ & ~& ~ & ~& ~& ~& ~\\ 
		\multicolumn{9}{^l^}{\linecharw[\odash]{\tablex\linewidth}}\\
	\end{retrotable}
	
	\begin{figure}[H]%
		\uheader{Resources}
		\subfloat{{	\begin{retrotable}[|L|l|][\tablesm\linewidth]
	\multicolumn{2}{_l_}{\linecharw[\_]{\tablesm\linewidth}}\\
	Resource & Amount  \\ \hline
	Gold &~ \\ \hline
	Iron &~ \\ \hline
	Leather &~ \\ \hline
	Lumber &~ \\ \hline
	Stone &~ \\ \hline
	~ &~ \\ \hline
	~ &~ \\ \hline
	~ &~ \\ \hline
	~ &~ \\ \hline
	~ &~ \\ \hline
	~ &~ \\ \hline
	~ &~ \\ \hline
	~ &~ \\ \hline
	~ &~ \\ \hline
	~ &~ \\ \hline
	~ &~ \\ \hline
	~ &~ \\ \hline
	~ &~ \\ \hline
	~ &~ \\ 
	\multicolumn{2}{^l^}{\linecharw[\odash]{\tablesm\linewidth}}\\
\end{retrotable} }}%
		\subfloat{{	\begin{retrotable}[|l|L|l|][\tablesm\linewidth]
	\multicolumn{3}{_l_}{\linecharw[\_]{\tablesm\linewidth}}\\
	Room & Resources & Rank  \\ \hline
	Great Hall & G \& S  &~ \\ \hline
	Armory & I \& Lu  &~ \\ \hline
	Bedrooms & Le \& G &~ \\ \hline
	Courtyard & S \& I &~ \\ \hline
	Kitchen & S \& Lu &~ \\ \hline
	Stables & Le \& Lu & ~ \\ \hline
	~ &~&~ \\ \hline
	~ &~&~ \\ \hline
	~ &~&~ \\ \hline
	~ &~&~ \\ \hline
	~ &~&~ \\ \hline
	~ &~&~ \\ \hline
	~ &~&~ \\ \hline
	~ &~&~ \\ \hline
	~ &~&~ \\ \hline
	~ &~&~ \\ \hline
	~ &~&~ \\ \hline
	~ &~&~ \\ \hline
	~ &~&~ \\ 
	\multicolumn{3}{^l^}{\linecharw[\odash]{\tablesm\linewidth}}\\
\end{retrotable} }}%
	\end{figure}
	
	\notes{10}{\textwidth}
	
	\notes{36}{\textwidth}
	
	\begin{tikzpicture}
		\draw[step=1cm,color=black,dashed] (-1,-1) grid (10,15);
	\end{tikzpicture}
	
		\begin{tikzpicture}
		\draw[step=1cm,color=black,dashed] (-1,-1) grid (10,15);
	\end{tikzpicture}
	

	
\end{document}