% !TeX program = xelatex
\documentclass[english,10pt,openany,a5paper]{book}
\usepackage{babel}
\usepackage{retrotype}
\usepackage{adjustbox}
\usepackage{layouts}
\usepackage{subfig}
\usepackage{afterpage}
 \usepackage[toc]{multitoc}
\usepackage{graphicx}
\usepackage{multicol}
\usepackage{afterpage}
\usepackage{makecell}
\usepackage{float}

%%%%%%%%%%%%%%%%%%%%%%%%%%%
\usepackage{fancyhdr} % Custom headers and footers
\fancypagestyle{fancyplain}{%
	\fancyhf{}% clear all header and footer fields
	\fancyfoot[C]{\thepage} % except the center
	\fancyhead[C]{\leftmark}
}

\pagestyle{fancyplain}
%%%%%%%%%%%%%%%%%%%%%%%%%%%

\usepackage{pgffor, ifthen}
\newcommand{\notes}[3][\empty]{%
	\noindent Notes\vspace{10pt}\\
	\foreach \n in {1,...,#2}{%
		\ifthenelse{\equal{#1}{\empty}}
		{\rule{#3}{0.5pt}\\}
		{\rule{#3}{0.5pt}\vspace{#1}\\}
	}
}

\newcommand{\tablex}{0.985}
\newcommand{\tablesm}{0.47}


\newcommand\blankpage{%
	\newpage 
	
	\thispagestyle{empty}%
	\
	
	\newpage }

\newcommand\tripip{\textbf{\op{\_}{/\textbackslash}} }

\newcommand\cirpip{\textbf{\op{O}{@}} }


% HYPERSETUP

\usepackage{hyperref}
\usepackage{bookmark}

\def\title{Rosethorn Keep}
\def\author{Charlotte Laskowski and Kaydevs}
\def\rtVer{1.0}
\def\rtDate{2024-06-15}

\hypersetup{%
	pdfborderstyle={/S/U/W 0},% underlined hyperlinks
	bookmarksdepth=5,% paragraph level depth bookmarks
	pdftitle={\title},
	pdfauthor={\author},
	%pdfkeywords={\keywords},
}

\smalltitles

\begin{document}
	
		\begin{titlepage}
	\begin{figure}%
		\centering
		\subfloat{\BVerbatimInput{img/rosethorn.txt}}%
	\end{figure}
	\begin{figure}%
		\centering
		\subfloat{{\BVerbatimInput{img/castle.txt} }}%
		\quad
		\subfloat{{\BVerbatimInput{img/keep.txt} }}%
	\end{figure}
	
	\begin{adjustbox}{center}
		\linechars[-]{7}
		{\hfill{}by \author{}}
		\linechars[-]{7}
	\end{adjustbox}
\end{titlepage}
	
	\blankpage
	
	\section{Credits}
	
	\thispagestyle{empty}
	
	Writing, layout: Charlotte Laskowski\\
	Writing: Kaydevs\\
	Editing: Jarrett Crader
	
	\skipline\skipline
	\skipline
	
	\href{https://selkie.itch.io/rosethorn-keep}{selkie.itch.io/rosethorn-keep}\\
	\href{https://dreamware.press}{dreamware.press}
	
	\newpage
	

	
	\toc[2]
	
	\thispagestyle{empty}
	
	
	\chapter{Tutorial}
	
	\thispagestyle{empty}
	
	
	\header{Materials}
	\tld  A sheet of paper and pencil, or something else to draw on.
	
	\tld  Some dice. A D6 will be required for the game. Online dice rollers are perfectly acceptable.
	
	\skipline
	\begin{adjustbox}{left}
		\em Reclaim what’s rightfully yours...\em
	\end{adjustbox}
	\begin{adjustbox}{right}
		\em Reclaim Rosethorn Keep...! \em
	\end{adjustbox}
	
	\section{Tutorial}

\tabpar[5]{\textbf{Objective:} Clear Rosethorn Keep}

Nestled deep in the forest, amongst ancient crypts and haunted woods, lies the legendary Rosethorn Keep. Largely in ruins, like any remains of civilization that was once around it, a significant amount of work is required to turn it into a flourishing castle once again. Despite its remote location, it will be a key stronghold in the fight against the Demon Lord.

\skipline

You have been tasked with recruiting adventurers, clearing the keep, and restoring it to its former glory. Do you have what it takes?

\skipline

Your first task, brave hero, is to recruit three companions to join you on your quest. These adventurers will clear out the dungeons, crypts, and other nefarious locations that surround the castle, and return with the Resources you need to rebuild it.

\skipline

\begin{adjustbox}{center}
	\textbf{Roll 1d6 three times, rerolling any duplicates.}
\end{adjustbox}

\skipline

Match the number on the die with the list of adventurers. These are the three companions you have successfully recruited.

\pagebreak

Their stat blocks are following. Notice that there are adventurers that you can’t use yet. This is as intended, and will be explained later.

\subsection{Companions}
Each companion has a \textbf{Trait}, a skill they can use once in every dungeon, and a number assigned to their \textbf{Stats}. If you go into a dungeon with two characters of the same Trait, any character can use that trait. It may only be used once per character with the trait.

\skipline

\begin{adjustbox}{center}
	\begin{retrotable}[~c|L|c~c~c~c~c|l~][1.0\linewidth]
		\# & Adventurer & ATK & DEF & MAG  & CHA & SPD & Stars\\ \Xhline{2\arrayrulewidth}
		1  & Archer          & 2 & 1 & 2 & 2 & 3 & *  \\
		\multicolumn{8}{l}{\linechars[-]{3} \emph{Trait:} Use SPD instead of DEF.}\\ \Xhline{2\arrayrulewidth}
		2  & Arriaga the Rogue          & 3 & 2 & 1 & 2 & 2 & *  \\
		\multicolumn{8}{l}{\linechars[-]{3} \emph{Trait:} Use ATK instead of MAG.}\\ \Xhline{2\arrayrulewidth}
		3  & Mathias the Squire        & 2 & 3 & 2 & 2 & 1 & * \\
		\multicolumn{8}{l}{\linechars[-]{3} \emph{Trait:} -1 from TN of next event.}\\ \Xhline{2\arrayrulewidth}
		4  & Chime the Bard            & 1 & 2 & 2 & 3 & 2 & * \\
		\multicolumn{8}{l}{\linechars[-]{3} \emph{Trait:} +1 to any stat.}\\ \Xhline{2\arrayrulewidth}
		5  & Ellen the Warrior           & 2 & 3 & 1 & 2 & 2 & * \\
		\multicolumn{8}{l}{\linechars[-]{3} \emph{Trait:} Use DEF instead of SPD.}\\ \Xhline{2\arrayrulewidth}
		6  & Lichen the Sling       & 3 & 1 & 2 & 2 & 2 & * \\
		\multicolumn{8}{l}{\linechars[-]{3} \emph{Trait:} +1 to any stat.}\\ \Xhline{2\arrayrulewidth}
		7  & Cassandra the Witch         & 1 & 1 & 3 & 3 & 2 & **\\
		\multicolumn{8}{l}{\linechars[-]{3} \emph{Trait:} Use MAG instead of ATK.}\\ \Xhline{2\arrayrulewidth}
		8  & Diego the Duelist         & 3 & 1 & 1 & 2 & 3 & ** \\
		\multicolumn{8}{l}{\linechars[-]{3} \emph{Trait:} Use SPD instead of DEF.}\\ \Xhline{2\arrayrulewidth}
		9  & Astrid, Shield Maiden   & 3 & 3 & 1 & 1 & 2 & **\\
		\multicolumn{8}{l}{\linechars[-]{3} \emph{Trait:} Use DEF instead of SPD.}\\ \Xhline{2\arrayrulewidth}
		10 & Eryn the Nun             & 1 & 2 & 3 & 3 & 1 & **\\
		\multicolumn{8}{l}{\linechars[-]{3} \emph{Trait:} -1 from TN of next event.}\\ \Xhline{2\arrayrulewidth}
		11 & Scarlet, Plague Doctor  & 1 & 3 & 4 & 1 & 1 & *** \\
		\multicolumn{8}{l}{\linechars[-]{3} \emph{Trait:} Use MAG instead of ATK.}\\ \Xhline{2\arrayrulewidth}
		12 & Gottfried the Knight           & 3 & 4 & 1 & 1 & 1 & *** \\
		\multicolumn{8}{l}{\linechars[-]{3} \emph{Trait:} Use ATK instead of MAG.}\\
	\end{retrotable}
\end{adjustbox}

\skipline



Now that you have recruited your companions and arrived at the Keep, a challenge glares back at you: This castle has been abandoned for many years. It must be fully secured and cleared of nefarious things before you can start anything else in your mission.

\begin{adjustbox}{center}
	\textbf{Roll 1d6 three times, and record each result.}
\end{adjustbox}
\begin{adjustbox}{center}
	\textbf{You will use these later.}
\end{adjustbox}

\skipline

Once you have rolled your dice, you start creeping through the halls and rooms of the Keep. Soon, you come across your first event...

\begin{adjustbox}{center}
	\textbf{Roll 1d6, and compare the result to the list of events.}
\end{adjustbox}

\subsection[Clearing the Keep]{Clearing the Keep}
\begin{tabbox}[3][3]
	
	\textbf{1. Stuck Door}\\
	
	\tabpar[1]{Door stuck! Please, I beg you, kick it down!}
	\tabpar[1]{\textbf{TN:} ATK 4 \cirpip}\skipline
	
	\textbf{2. Derelict Armory}\\
	
	\tabpar[1]{Old weapons hang precariously from the wall.}
	\tabpar[1]{\textbf{TN:} DEF 4 \cirpip}\skipline
	
	\textbf{3. Haunted Painting}\\
	
	\tabpar[1]{Feels like if you jumped on it, you would go somewhere far away...}
	\tabpar[1]{\textbf{TN:} MAG 5 \cirpip}\skipline
	
	\textbf{4. Castle Ghost}\\
	
	\tabpar[1]{Doesn’t seem aggressive, maybe you can ask it for directions.}
	\tabpar[1]{\textbf{TN:} CHA 5 \cirpip}\skipline
	
	\textbf{5. Doorway Trap}\\
	
	\tabpar[1]{A brick placed on top of an ajar door.}
	\tabpar[1]{\textbf{TN:} SPD 5 \cirpip}\skipline
	
	\textbf{6. Living Armor}\\
	
	\tabpar[1]{A living suit of armor, guarding the keep from intruders.}
	\tabpar[1]{\textbf{TN:} ATK 6 \tripip}\skipline
\end{tabbox}
\skipline

Now you must complete each event, for better or worse – take any of the numbers you rolled and recorded previously, and choose a character with a suitable stat for the event. Add the recorded die result and the companion’s stat together. If the total meets or beats the TN, you have successfully completed the event! If you aren’t able to succeed, check the \textbf{Traits} of your companions.

\skipline

Using traits is a great way to succeed at an event you might not have been able to otherwise. Each companion can be used twice in a dungeon, so you may run through some without using some of them at all! However, if a companion is defeated, they may not be used again, and must rest in the castle before they can go out once more.

\skipline

After you have completed your first event, roll again, and complete the second one. If you roll an event that is duplicate to one you have rolled before, you automatically succeed, but in later dungeons, this will not count towards your resource gain. You complete as many events, one at a time, as companions you have brought with you. For the first dungeon this is three.

\skipline
Even if you don’t succeed in every event, once you have completed all three, the Keep is yours.

\skipline

\begin{adjustbox}{center}
	\textbf{CONGRATULATIONS!}
\end{adjustbox}
\begin{adjustbox}{center}
	You have earned: \emph{Rosethorn Keep}
\end{adjustbox}
\begin{adjustbox}{center}
	Your \textbf{Great Hall} has been rebuilt to Tier 1.
\end{adjustbox}

\skipline

Wait... Your Great Hall? What’s that?

\skipline

Your castle has six sections, each providing a different benefit for your companions, and requiring a different amount of Resources to build and upgrade. When you cleared the castle, you found that the Great Hall was still standing, and in decent condition. (More about the castle is found on the next page.)

\skipline

The rest of the castle is in shambles, and will need Resources to restore it to its former glory. 
To collect Resources, you delve into dungeons and overcome the events within. The type of dungeon you choose determines what Resources you get, and the Resources you stockpile in turn determine what parts of the castle you can rebuild and upgrade. Each event in a dungeon is marked by a \tripip  (triangle) or a \cirpip (circle). The \tripip marks a combat encounter, and the \cirpip marks a non-combat encounter.
	
	\chapter{The Castle}

\thispagestyle{empty}

\begin{tabbox}[4][4]
	\subsubsection{Great Hall}
	\tabpar[3]{Determines the maximum tier you can upgrade other sections to.}
	\tabpar[3]{\textbf{Resources:} Gold \& Stone}
	
	\subsubsection{Armory}
	\tabpar[3]{-1 per tier from the TN of \tripip Combat events.}
	\tabpar[3]{\textbf{Resources:} Iron \& Lumber}
	
	\subsubsection{Courtyard}
	\tabpar[3]{-1 per tier from the TN of \cirpip Noncombat events.}
	\tabpar[3]{\textbf{Resources:} Stone \& Iron}
	
	\subsubsection{Kitchen}
	\tabpar[3]{Choose a number of dice equal to tier to reroll at dungeon start.}
	\tabpar[3]{\textbf{Resources:} Stone \& Lumber}
	
	\subsubsection{Stables}
	\tabpar[3]{Bring back a number of Resources from the dungeon equal to tier.}
	\tabpar[3]{\textbf{Resources:} Lumber \& Leather}
	
	\subsubsection{Bedrooms}
	\tabpar[3]{Regenerate more companions for the next dungeon, equal to tier + 1.}
	\tabpar[3]{\textbf{Resources:} Leather \& Gold}
\end{tabbox}

\pagebreak

\subsection{Great Hall}

Determines the maximum tier you can upgrade other sections to.
\begin{tabbox}[4][4]
	
\textbf{Tier 0}\\

	\tabpar[3]{N/A}
	\tabpar[3]{\textbf{Upgrade Cost:} N/A}\skipline

\textbf{Tier 1}\\

	\tabpar[3]{Buildings may be up to Tier 1.}
	\tabpar[3]{\textbf{Upgrade Cost:} N/A}\skipline
	
\textbf{Tier 2}\\

	\tabpar[3]{Buildings may be up to Tier 2.}
	\tabpar[3]{Roll 1d6 to determine your next companion.}
	\tabpar[3]{\textbf{Upgrade Cost:} 10 Gold \& 5 Stone}\skipline
	
\textbf{Tier 3}\\

	\tabpar[3]{Buildings may be up to Tier 3.}
	\tabpar[3]{Roll 2d6 to determine your next companion.}
	\tabpar[3]{\textbf{Upgrade Cost:} 15 Gold \& 10 Stone}\skipline
	
\textbf{Tier 4}\\

	\tabpar[3]{Buildings may be up to Tier 4.}
	\tabpar[3]{Roll 3d6 to determine your next companion.}
	\tabpar[3]{\textbf{Upgrade Cost:} 20 Gold \& 15 Stone}\skipline
	
\textbf{Tier 5}\\

	\tabpar[3]{Buildings may be up to Tier 5.}
	\tabpar[3]{Roll 3d6 and drop the lowest result to determine your next companion.}
	\tabpar[3]{\textbf{Upgrade Cost:} 30 Gold \& 30 Stone}\skipline
\end{tabbox}

\pagebreak


\subsection{Armory}

-1 per tier from the TN of \tripip Combat events.
\skipline

Example: You roll a 4 for the event while exploring the Crypt Dungeon, and come across a skeleton. The TN for this encounter is 4, plus the tier of your Great Hall. If your Great Hall is tier 2, the TN of the encounter. When including the Armory’s benefit, the final TN of that encounter is 4.

\begin{tabbox}[4][4]
	
	\textbf{Tier 0}\\
	
	\tabpar[3]{N/A}
	\tabpar[3]{\textbf{Upgrade Cost:} N/A}\skipline
	
	\textbf{Tier 1}\\
	
	\tabpar[3]{-1 from the TN of \tripip Combat events.}
	\tabpar[3]{\textbf{Upgrade Cost:} 5 Iron \& 1 Lumber}\skipline
	
	\textbf{Tier 2}\\
	
	\tabpar[3]{-1 from the TN of \tripip Combat events.}
	\tabpar[3]{\textbf{Upgrade Cost:} 10 Iron \& 5 Lumber}\skipline
	
	\textbf{Tier 3}\\
	
	\tabpar[3]{-1 from the TN of \tripip Combat events.}
	\tabpar[3]{\textbf{Upgrade Cost:} 15 Iron \& 10 Lumber}\skipline
	
	\textbf{Tier 4}\\
	
	\tabpar[3]{-1 from the TN of \tripip Combat events.}
	\tabpar[3]{\textbf{Upgrade Cost:} 20 Iron \& 15 Lumber}\skipline
	
	\textbf{Tier 5}\\
	
	\tabpar[3]{-1 from the TN of \tripip Combat events.}
	\tabpar[3]{\textbf{Upgrade Cost:} 30 Iron \& 30 Lumber}\skipline
\end{tabbox}

\pagebreak


\subsection{Courtyard}

-1 per tier from the TN of \cirpip Noncombat events.
\skipline

Example: You roll a 4 for the event while exploring the Monster Lair, and come across a hunting dog. The TN for this encounter is 4, plus the tier of your Great Hall. If your Great Hall is tier 2, the TN of the encounter. When including the Courtyard’s benefit, the final TN of that encounter is 4.

\skipline

\begin{tabbox}[4][4]
	
	\textbf{Tier 0}\\
	
	\tabpar[3]{N/A}
	\tabpar[3]{\textbf{Upgrade Cost:} N/A}\skipline
	
	\textbf{Tier 1}\\
	
	\tabpar[3]{-1 from the TN of \cirpip Noncombat events.}
	\tabpar[3]{\textbf{Upgrade Cost:} 5 Stone \& 1 Iron}\skipline
	
	\textbf{Tier 2}\\
	
	\tabpar[3]{-1 from the TN of \cirpip Noncombat events.}
	\tabpar[3]{\textbf{Upgrade Cost:} 10 Stone \& 5 Iron}\skipline
	
	\textbf{Tier 3}\\
	
	\tabpar[3]{-1 from the TN of \cirpip Noncombat events.}
	\tabpar[3]{\textbf{Upgrade Cost:} 15 Stone \& 10 Iron}\skipline
	
	\textbf{Tier 4}\\
	
	\tabpar[3]{-1 from the TN of \cirpip Noncombat events.}
	\tabpar[3]{\textbf{Upgrade Cost:} 20 Stone \& 15 Iron}\skipline
	
	\textbf{Tier 5}\\
	
	\tabpar[3]{-1 from the TN of \cirpip Noncombat events.}
	\tabpar[3]{\textbf{Upgrade Cost:} 30 Stone \& 30 Iron}\skipline
\end{tabbox}

\pagebreak


\subsection{Kitchen}

Choose a number of recorded dice to reroll at dungeon start.
\skipline

\begin{tabbox}[4][4]
	
	\textbf{Tier 0}\\
	
	\tabpar[3]{N/A}
	\tabpar[3]{\textbf{Upgrade Cost:} N/A}\skipline
	
	\textbf{Tier 1}\\
	
	\tabpar[3]{Reroll up to 1 die at the beginning of a dungeon, only once per die.}
	\tabpar[3]{\textbf{Upgrade Cost:} 5 Stone \& 1 Lumber}\skipline
	
	\textbf{Tier 2}\\
	
	\tabpar[3]{Reroll up to 2 dice at the beginning of a dungeon, only once per die.}
	\tabpar[3]{\textbf{Upgrade Cost:} 10 Stone \& 5 Lumber}\skipline
	
	\textbf{Tier 3}\\
	
	\tabpar[3]{Reroll up to 3 dice at the beginning of a dungeon, only once per die.}
	\tabpar[3]{\textbf{Upgrade Cost:} 15 Stone \& 10 Lumber}\skipline
	
	\textbf{Tier 4}\\
	
	\tabpar[3]{Reroll up to 4 dice at the beginning of a dungeon, only once per die.}
	\tabpar[3]{\textbf{Upgrade Cost:} 20 Stone \& 15 Lumber}\skipline
	
	\textbf{Tier 5}\\
	
	\tabpar[3]{Reroll up to 5 dice at the beginning of a dungeon, only once per die.}
	\tabpar[3]{\textbf{Upgrade Cost:} 30 Stone \& 30 Lumber}\skipline
\end{tabbox}

\pagebreak


\subsection{Stables}

Bring back extra Resources.
\skipline

\begin{tabbox}[4][4]
	
	\textbf{Tier 0}\\
	
	\tabpar[3]{N/A}
	\tabpar[3]{\textbf{Upgrade Cost:} N/A}\skipline
	
	\textbf{Tier 1}\\
	
	\tabpar[3]{Bring back 1 more of each resource gained.}
	\tabpar[3]{\textbf{Upgrade Cost:} 5 Lumber \& 1 Leather}\skipline
	
	\textbf{Tier 2}\\
	
	\tabpar[3]{Bring back 2 more of each resource gained.}
	\tabpar[3]{\textbf{Upgrade Cost:} 10 Lumber \& 5 Leather}\skipline
	
	\textbf{Tier 3}\\
	
	\tabpar[3]{Bring back 3 more of each resource gained.}
	\tabpar[3]{\textbf{Upgrade Cost:} 15 Lumber \& 10 Leather}\skipline
	
	\textbf{Tier 4}\\
	
	\tabpar[3]{Bring back 4 more of each resource gained.}
	\tabpar[3]{\textbf{Upgrade Cost:} 20 Lumber \& 15 Leather}\skipline
	
	\textbf{Tier 5}\\
	
	\tabpar[3]{Bring back 5 more of each resource gained.}
	\tabpar[3]{\textbf{Upgrade Cost:} 30 Lumber \& 30 Leather}\skipline
\end{tabbox}

\pagebreak


\subsection{Bedrooms}

Regenerate more companions between dungeons after they are defeated.
\skipline

\begin{tabbox}[4][4]
	
	\textbf{Tier 0}\\
	
	\tabpar[3]{Regenerate 1 companion between dungeons.}
	\tabpar[3]{\textbf{Upgrade Cost:} N/A}\skipline
	
	\textbf{Tier 1}\\
	
	\tabpar[3]{Regenerate 2 companions between dungeons.}
	\tabpar[3]{\textbf{Upgrade Cost:} 5 Leather \& 1 Gold}\skipline
	
	\textbf{Tier 2}\\
	
	\tabpar[3]{Regenerate 3 companions between dungeons.}
	\tabpar[3]{\textbf{Upgrade Cost:} 10 Leather \& 5 Gold}\skipline
	
	\textbf{Tier 3}\\
	
	\tabpar[3]{Regenerate 4 companions between dungeons.}
	\tabpar[3]{\textbf{Upgrade Cost:} 15 Leather \& 10 Gold}\skipline
	
	\textbf{Tier 4}\\
	
	\tabpar[3]{Regenerate 5 companions between dungeons.}
	\tabpar[3]{\textbf{Upgrade Cost:} 20 Leather \& 15 Gold}\skipline
	
	\textbf{Tier 5}\\
	
	\tabpar[3]{Regenerate 6 companions between dungeons.}
	\tabpar[3]{\textbf{Upgrade Cost:} 30 Leather \& 30 Gold}\skipline
\end{tabbox}

\pagebreak
	
	
	\chapter{Dungeons}

\thispagestyle{empty}

To calculate the total Resources you bring back to the castle, multiply the dungeon’s Base Resources by the number of events completed successfully. Remember to add the tier of your Stables to both Resources before multiplying by the amount of successful events.

\skipline

At Tiers 2 \& 3, roll \textbf{2d6} per companion to determine the event. At Tiers 4 \& 5, roll \textbf{3d6} per companion to determine the event.

\skipline

If you bring more than five companions, you also fight the Boss. This is automatic and replaces one of the dungeon’s events. To defeat bosses, companions must overcome both TNs listed.

\begin{multicols}{2}

	\subsubsection{Abandoned? Mine}
	\tabpar[1]{Though taken over by a beast of sulfur, the materials needed for your forge lie within.}
	\tabpar[1]{\textbf{Resources:} 5 Iron \& 3 Stone}
	
	\subsubsection{Crypt}
	\tabpar[1]{An old tomb, now overrun by undead. Within, many treasures await.}
	\tabpar[1]{\textbf{Resources:} 5 Gold \& 3 Iron}
	
	\columnbreak
	
	\subsubsection{Haunted Forest}
	\tabpar[1]{Cursed by an ancient evil, and yet still home to bandits.}
	\tabpar[1]{\textbf{Resources:} 5 Lumber \& 3 Leather}
	
	\subsubsection{Monster Lair}
	\tabpar[1]{The lair of a great white beast. Holds the treasures of misfortune seekers.}
	\tabpar[1]{\textbf{Resources:} 5 Leather \& 3 Gold}
	
\end{multicols}
	
	\subsubsection{Ruins}
	\tabpar[1]{There are many abandoned towns near the castle. Old secrets hide here.}
	\tabpar[1]{\textbf{Resources:} 5 Stone \& 3 Lumber}


\pagebreak

\subsection{Abandoned? Mine}

\begin{multicols}{2}
		\textbf{1. Drunk Miners}\\
		
		\tabpar[1]{They won’t let you pass unless you can out-drink them.}
		\tabpar[1]{\textbf{TN:} DEF 3 + [TIER] \cirpip}\skipline
		
		\textbf{2. Lost Miner}\\
		
		\tabpar[1]{Convince him you mean no harm, and guide him to safety.}
		\tabpar[1]{\textbf{TN:} CHA 3 + [TIER] \cirpip}\skipline
		
		\textbf{3. Tunnel Collapse}\\
		
		\tabpar[1]{Run away before you are crushed like my hopes and dreams.}
		\tabpar[1]{\textbf{TN:} SPD 4 + [TIER] \cirpip}\skipline
		
		\textbf{4. Toxic Gases}\\
		
		\tabpar[1]{You should’ve brought a canary... and shouldn't have dressed in yellow.}
		\tabpar[1]{\textbf{TN:} DEF 4 + [TIER] \cirpip}\skipline
		
		\textbf{5. Flooded Shaft}\\
		
		\tabpar[1]{Hold your breath and look for goodies in the depths.}
		\tabpar[1]{\textbf{TN:} SPD 5 + [TIER] \cirpip}\skipline
		
		\textbf{6. Cave Troll}\\
		
		\tabpar[1]{Came down here to hide from the sunlight and got lost.}
		\tabpar[1]{\textbf{TN:} ATK 6 + [TIER] \tripip}\skipline
		
		\columnbreak
		
		\textbf{7. Kobolds}\\
		
		\tabpar[1]{A hunting party looking for food to eat, and you look really tasty.}
		\tabpar[1]{\textbf{TN:} ATK 5 + [TIER] \tripip}\skipline
		
		\textbf{8. Cave Paintings}\\
		
		\tabpar[1]{Either drawings of animals, or long forgotten magics.}
		\tabpar[1]{\textbf{TN:} MAG 5 + [TIER] \cirpip}\skipline
		
		\textbf{9. Cave Shrooms}\\
		
		\tabpar[1]{Definitely \so{(Not)} safe to eat \so{? (!)}}
		\tabpar[1]{\textbf{TN:} DEF 5 + [TIER] \cirpip}\skipline
		
		\textbf{10. Glowing Rocks}\\
		
		\tabpar[1]{Valuable magical rocks, do not touch with bare hands.}
		\tabpar[1]{\textbf{TN:} MAG 5 + [TIER] \cirpip}\skipline
		
		\textbf{11. Ore Vein}\\
		
		\tabpar[1]{It will take a lot of work to get it out without proper equipment.}
		\tabpar[1]{\textbf{TN:} ATK 6 + [TIER] \cirpip}\skipline
		
		\textbf{12. Underground Village}\\
		
		\tabpar[1]{Lost to time, you can trade light for iron and stone.}
		\tabpar[1]{\textbf{TN:} CHA 7 + [TIER] \cirpip}\skipline
		
		\columnbreak
		
		\textbf{13. Bat Swarm}\\
		
		\tabpar[1]{Run away or you’ll drown in bats!}
		\tabpar[1]{\textbf{TN:} SPD 7 + [TIER] \tripip}\skipline
		
		\textbf{14. Tight Tunnel}\\
		
		\tabpar[1]{Very claustrophobic, no one would willingly enter there. Unless...}
		\tabpar[1]{\textbf{TN:} SPD 6 + [TIER] \cirpip}\skipline
		
		\textbf{15. Underground River}\\
		
		\tabpar[1]{Bottomless and eerily cold waters, just hope it’s not the Styx.}
		\tabpar[1]{\textbf{TN:} DEF 6 + [TIER] \cirpip}\skipline
		
		\columnbreak
		
		\textbf{16. Giant Bat}\\
		
		\tabpar[1]{A creature of the night, its echolocation can kill.}
		\tabpar[1]{\textbf{TN:} ATK 6 + [TIER] \tripip}\skipline
		
		\textbf{17. Earthquake}\\
		
		\tabpar[1]{The earth rumbles in anger at your intrusion to the underground world.}
		\tabpar[1]{\textbf{TN:} DEF 7 + [TIER] \cirpip}\skipline
		
		\textbf{18. Lava Flow}\\
		
		\tabpar[1]{The veins of the earth open before you, and spew molten blood.}
		\tabpar[1]{\textbf{TN:} SPD 8 + [TIER] \cirpip}\skipline
	
\end{multicols}
\textbf{BOSS: Dragon} -- Strong, smart, and prone to vanity, praise and a hidden blade will do the job.\\
\textbf{TN:} CHA 11 \tripip \& ATK 12 \tripip

\pagebreak

\subsection{Crypt}
\begin{multicols}{2}

		\textbf{1. Undertaker}\\
		
		\tabpar[1]{A lone undertaker who hasn't realized the crypt is overrun.}
		\tabpar[1]{\textbf{TN:} CHA 3 + [TIER] \cirpip}\skipline
		
		\textbf{2. Illusory Wall}\\
		
		\tabpar[1]{You sense an illusory wall, try to find it without activating it.}
		\tabpar[1]{\textbf{TN:} MAG 3 + [TIER] \cirpip}\skipline
		
		\textbf{3. Arrow Trap}\\
		
		\tabpar[1]{You stepped on a pressure plate, disarm the trap or be skewered.}
		\tabpar[1]{\textbf{TN:} SPD 4 + [TIER] \cirpip}\skipline
		
		\textbf{4. Skeleton}\\
		
		\tabpar[1]{It rises from its tomb and starts rattling at you.}
		\tabpar[1]{\textbf{TN:} ATK 4 + [TIER] \tripip}\skipline
		
		\textbf{5. Cobwebs}\\
		
		\tabpar[1]{You’re trapped in a hallway filled with cobwebs from top to bottom.}
		\tabpar[1]{\textbf{TN:} DEF 5 + [TIER] \cirpip}\skipline
		
		\columnbreak
		
		\textbf{6. Ghost}\\
		
		\tabpar[1]{If you remind a ghost that it is already dead, it will pass away.}
		\tabpar[1]{\textbf{TN:} CHA 6 + [TIER] \cirpip}\skipline
		
		\textbf{7. Slime}\\
		
		\tabpar[1]{You touch something wet, sticky, and trying to digest you.}
		\tabpar[1]{\textbf{TN:} DEF 5 + [TIER] \tripip}\skipline
		
		\textbf{8. Stone Coffin}\\
		
		\tabpar[1]{Is treasure worth unearthing what should remain buried?}
		\tabpar[1]{\textbf{TN:} MAG 5 + [TIER] \cirpip}\skipline
		
		\textbf{9. Gargoyle}\\
		
		\tabpar[1]{Meant to guard from evil spirits. You look like evil spirits.}
		\tabpar[1]{\textbf{TN:} ATK 5 + [TIER] \tripip}\skipline
		
		
		\textbf{10. Mimic}\\
		
		\tabpar[1]{Roll 1d6. If you have an equivalent die, steal its treasure.}
		\skipline
		
		\columnbreak
		
		\textbf{11. Banshee}\\
		
		\tabpar[1]{A spirit of bad omens. Run away before you find out who dies.}
		\tabpar[1]{\textbf{TN:} SPD 6 + [TIER] \tripip}\skipline
		
		\textbf{12. Mysterious Altar}\\
		
		\tabpar[1]{A place of dark rituals? Or a long forgotten worship site?}
		\tabpar[1]{\textbf{TN:} MAG 7 + [TIER] \cirpip}\skipline
		
		\textbf{13. Snake Pit}\\
		
		\tabpar[1]{Snakes, why did it have to be snakes?}
		\tabpar[1]{\textbf{TN:} DEF 6 + [TIER] \cirpip}\skipline
		
		\textbf{14. "Bottomless" Pit}\\
		
		\tabpar[1]{You can see the bottom.}
		\tabpar[1]{\textbf{TN:} SPD 6 + [TIER] \cirpip}\skipline
		
		\columnbreak
		
		\textbf{15. Descending Ceiling}\\
		
		\tabpar[1]{Hold the ceiling up until someone can deactivate the trap.}
		\tabpar[1]{\textbf{TN:} DEF 6 + [TIER] \cirpip}\skipline
		
		\textbf{16. Ghoul}\\
		
		\tabpar[1]{A man-eating undead.}
		\tabpar[1]{\textbf{TN:} ATK 6 + [TIER] \tripip}\skipline
		
		\textbf{17. Talking Statue}\\
		
		\tabpar[1]{If you solve the riddle it recites, you might find treasure.}
		\tabpar[1]{\textbf{TN:} CHA 7 + [TIER] \cirpip}\skipline
		
		\textbf{18. Rolling Ball Trap}\\
		
		\tabpar[1]{You should have known that treasure was too good to be true.}
		\tabpar[1]{\textbf{TN:} SPD 8 + [TIER] \cirpip}\skipline
		

\end{multicols}

\textbf{BOSS: Necromancer} -- Has taken over the crypt, and will only leave if bested by wits.\\
\textbf{TN:} MAG 11 \tripip \& CHA 12 \tripip

\pagebreak

\subsection{Haunted Forest}

\begin{multicols}{2}

		\textbf{1. Forest Clearing}\\
		
		\tabpar[1]{\textbf{A mysterious clearing that gives you goosebumps.}}
		\tabpar[1]{\textbf{TN:} MAG 3 + [TIER] \cirpip}\skipline
		
		\textbf{2. Snare Trap}\\
		
		\tabpar[1]{Left behind by poachers. Not dangerous unless it’s to your ego.}
		\tabpar[1]{\textbf{TN:} SPD 3 + [TIER] \cirpip}\skipline
		
		\textbf{3. Man-Eater Plant}\\
		
		\tabpar[1]{True to its name.}
		\tabpar[1]{\textbf{TN:} ATK 4 + [TIER] \tripip}\skipline
		
		\textbf{4. Swinging Log}\\
		
		\tabpar[1]{Another poacher’s trap. This one will be a headache.}
		\tabpar[1]{\textbf{TN:} SPD 4 + [TIER] \cirpip}\skipline
		
		\textbf{5. Wolf Pack}\\
		
		\tabpar[1]{The evil forces in this wood have brought this wolf pack to hunt you.}
		\tabpar[1]{\textbf{TN:} ATK 5 + [TIER] \cirpip}\skipline
		
		\textbf{6. Laurel Hell}\\
		
		\tabpar[1]{If you get in, you can’t get out.}
		\tabpar[1]{\textbf{TN:} DEF 6 + [TIER] \tripip}\skipline
		
		\columnbreak
		
		\textbf{7. Treant}\\
		
		\tabpar[1]{A living, talking tree who can be reasoned with.}
		\tabpar[1]{\textbf{TN:} CHA 5 + [TIER] \cirpip}\skipline
		
		\textbf{8. Pitcher Plant}\\
		
		\tabpar[1]{It swallows you whole. Resist the acid until you can escape.}
		\tabpar[1]{\textbf{TN:} MAG 5 + [TIER] \tripip}\skipline
		
		\textbf{9. Walking Shroom}\\
		
		\tabpar[1]{A common forest creature. Edible in a pinch, but questionable.}
		\tabpar[1]{\textbf{TN:} ATK 5 + [TIER] \tripip}\skipline
		
		\textbf{10. Old Druid}\\
		
		\tabpar[1]{Here to drive out the evil, but not yet convinced you’re here to help.}
		\tabpar[1]{\textbf{TN:} CHA 5 + [TIER] \cirpip}\skipline
		
		\textbf{11. The Silence}\\
		
		\tabpar[1]{All noise leaves the forest, a sign of danger beyond your abilities.}
		\tabpar[1]{\textbf{TN:} SPD 6 + [TIER] \tripip}\skipline
		
		\columnbreak
		
		\textbf{12. Aquelarre}\\
		
		\tabpar[1]{A witches’ reunion. Partake, and they will let you leave.}
		\tabpar[1]{\textbf{TN:} MAG 7 + [TIER] \cirpip}\skipline
		
		
		\textbf{13. Forest Spirit}\\
		
		\tabpar[1]{A lost forest spirit, looking for a source of life to sap.}
		\tabpar[1]{\textbf{TN:} DEF 6 + [TIER] \tripip}\skipline
		
		\textbf{14. Dryad}\\
		
		\tabpar[1]{Be prepared lest you be enthralled or turned into a plant.}
		\tabpar[1]{\textbf{TN:} MAG 6 + [TIER] \cirpip}\skipline
		
		\textbf{15. Poacher's Camp}\\
		
		\tabpar[1]{They’re hiding from someone, and will fight to keep hidden.}
		\tabpar[1]{\textbf{TN:} ATK 6 + [TIER] \tripip}\skipline
		
		\columnbreak
		
		\textbf{16. Elven Rangers}\\
		
		\tabpar[1]{They are looking for someone, and must be convinced it’s not you.}
		\tabpar[1]{\textbf{TN:} CHA 6 + [TIER] \cirpip}\skipline
		
		\textbf{17. Deep Fog}\\
		
		\tabpar[1]{You can’t see your hand held out before you. Stay on guard.}
		\tabpar[1]{\textbf{TN:} DEF 7 + [TIER] \tripip}\skipline
		
		\textbf{18. Fairy Ring}\\
		
		\tabpar[1]{You stepped into a fairy ring. Step away before you are taken.}
		\tabpar[1]{\textbf{TN:} SPD 8 + [TIER] \cirpip}\skipline
		
\end{multicols}

\textbf{BOSS: Cursed Tree} -- A host of hexes, fight it so you may prevent evil from spreading.\\
\textbf{TN:} ATK 11 \tripip \& MAG 12 \tripip

\pagebreak

\subsection{Monster Lair}

\begin{multicols}{2}
		
		\textbf{1. Hoard}\\
		
		\tabpar[1]{Treasure! You might find some neat stuff inside if its not enchanted.}
		\tabpar[1]{\textbf{TN:} MAG 3 + [TIER] \cirpip}\skipline
		
		\textbf{2. Fresh Tracks}\\
		
		\tabpar[1]{Follow the tracks if you are fast enough to catch who made them.}
		\tabpar[1]{\textbf{TN:} SPD 4 + [TIER] \cirpip}\skipline
		
		\textbf{3. Pile of Bones}\\
		
		\tabpar[1]{Search the bones for remaining valuables if you can stomach it.}
		\tabpar[1]{\textbf{TN:} DEF 4 + [TIER] \cirpip}\skipline
		
		\textbf{4. Lost Hound}\\
		
		\tabpar[1]{A lost hunting dog. You think its owner came monster hunting.}
		\tabpar[1]{\textbf{TN:} SPD 4 + [TIER] \cirpip}\skipline
		
		\textbf{5. Giant Rat}\\
		
		\tabpar[1]{He makes all of the rules.}
		\tabpar[1]{\textbf{TN:} ATK 5 + [TIER] \tripip}\skipline
		
		\textbf{6. Ambush}\\
		
		\tabpar[1]{Ambushed by an unknown foe, fight back the best you can.}
		\tabpar[1]{\textbf{TN:} ATK 6 + [TIER] \tripip}\skipline
		
		\columnbreak
		
		\textbf{7. Bear Trap}\\
		
		\tabpar[1]{Many bear traps lie between you and your destination.}
		\tabpar[1]{\textbf{TN:} SPD 5 + [TIER] \cirpip}\skipline
		
		\textbf{8. Signs}\\
		
		\tabpar[1]{Marked with warnings, interpret them to learn what’s ahead.}
		\tabpar[1]{\textbf{TN:} MAG 5 + [TIER] \cirpip}\skipline
		
		\textbf{9. Cave-in}\\
		
		\tabpar[1]{The lair’s ceiling is caving in. Get out before you’re crushed.}
		\tabpar[1]{\textbf{TN:} SPD 5 + [TIER] \cirpip}\skipline
		
		\textbf{10. Knightslayer Snail}\\
		
		\tabpar[1]{The mortal enemy of Knights and Squires: the snail.}
		\tabpar[1]{\textbf{TN:} DEF 5 + [TIER] \tripip}\skipline
		
		\textbf{11. Summoning Circle}\\
		
		\tabpar[1]{An incomplete summoning circle. You may be able to finish the ritual...}
		\tabpar[1]{\textbf{TN:} MAG 6 + [TIER] \cirpip}\skipline
		
		\textbf{12. Monster Hunter}\\
		
		\tabpar[1]{A hunter missing its hound. Have you seen it?}
		\tabpar[1]{\textbf{TN:} CHA 7 + [TIER] \cirpip}\skipline
		
		\columnbreak
		
		\textbf{13. Harpies}\\
		
		\tabpar[1]{They made a nest here, and you seem like easy prey.}
		\tabpar[1]{\textbf{TN:} ATK 6 + [TIER] \tripip}\skipline
		
		\textbf{14. Stalking Beast}\\
		
		\tabpar[1]{An unknown monster is looking for you, and you can’t fight it. Hide!}
		\tabpar[1]{\textbf{TN:} SPD 6 + [TIER] \tripip}\skipline
		
		\textbf{15. Cockatrice}\\
		
		\tabpar[1]{Be mindful of its petrifying gaze. You must counter its magic.}
		\tabpar[1]{\textbf{TN:} MAG 6 + [TIER] \tripip}\skipline
		
		\columnbreak
		
		\textbf{16. Corpse Pit}\\
		
		\tabpar[1]{Remains of meals from the boss, but you see the glint of gold...}
		\tabpar[1]{\textbf{TN:} DEF 6 + [TIER] \cirpip}\skipline
		
		\textbf{17. Loot Pile}\\
		
		\tabpar[1]{A glinting pile of gold. Try to take some, but be quick.}
		\tabpar[1]{\textbf{TN:} SPD 7 + [TIER] \cirpip}\skipline
		
		\textbf{18. Ogre}\\
		
		\tabpar[1]{A great man armed with a heavy club and an insatiable hunger.}
		\tabpar[1]{\textbf{TN:} ATK 8 + [TIER] \cirpip}\skipline
\end{multicols}
\textbf{BOSS: Rabbit} -- An unassuming but lethal beast that cannot be slayed by you or any.\\
\textbf{TN:} DEF 11 \tripip \& SPD 12 \cirpip

\pagebreak

\subsection{Ruins}

\begin{multicols}{2}
		
		\textbf{1. Old Cellar}\\
		
		\tabpar[1]{May contain treasures left behind, or rotting food.}
		\tabpar[1]{\textbf{TN:} DEF 3 + [TIER] \cirpip}\skipline
		
		\textbf{2. Hermit}\\
		
		\tabpar[1]{He’s lived alone for decades. Who knows what secrets he holds...}
		\tabpar[1]{\textbf{TN:} CHA 3 + [TIER] \cirpip}\skipline
		
		\textbf{3. Pack of Stray Dogs}\\
		
		\tabpar[1]{Left behind by their owners, now you’re in their territory.}
		\tabpar[1]{\textbf{TN:} SPD 4 + [TIER] \tripip}\skipline
		
		\textbf{4. Crumbling Room}\\
		
		\tabpar[1]{An old bedroom that falls apart when you enter.}
		\tabpar[1]{\textbf{TN:} SPD 4 + [TIER] \cirpip}\skipline
		
		\textbf{5. Bookshelf}\\
		
		\tabpar[1]{You can’t tell if it contains old grimoires or romance novels.}
		\tabpar[1]{\textbf{TN:} MAG 5 + [TIER] \cirpip}\skipline
		
		\textbf{6. Bandit}\\
		
		\tabpar[1]{A bandit looking to score some cash. You have some cash.}
		\tabpar[1]{\textbf{TN:} ATK 6 + [TIER] \tripip}\skipline
		
		\columnbreak
		
		\textbf{7. Tainted Well}\\
		
		\tabpar[1]{It sickens you when you take a sip. Resist the blight.}
		\tabpar[1]{\textbf{TN:} DEF 5 + [TIER] \cirpip}\skipline
		
		\textbf{8. Broken Windmill}\\
		
		\tabpar[1]{Someone must have confused it for a giant.}
		\tabpar[1]{\textbf{TN:} MAG 5 + [TIER] \cirpip}\skipline
		
		\textbf{9. Mimic}\\
		
		\tabpar[1]{Roll 1d6. If you have an equivalent die, steal its treasure.}\skipline
		
		\textbf{10. Dead Horse}\\
		
		\tabpar[1]{Maybe if you beat it enough, something interesting will happen.}
		\tabpar[1]{\textbf{TN:} ATK 5 + [TIER] \cirpip}\skipline
		
		\textbf{11. Hidden Room}\\
		
		\tabpar[1]{Equal chance of accessing it as there is of activating a trap.}
		\tabpar[1]{\textbf{TN:} SPD 6 + [TIER] \cirpip}\skipline
		
		\textbf{12. Bridge Troll}\\
		
		\tabpar[1]{Trying to extort you. Luckily, all he wants is coin.}
		\tabpar[1]{\textbf{TN:} ATK 7 + [TIER] \tripip}\skipline
		
		\columnbreak
		
		\textbf{13. Doppelganger}\\
		
		\tabpar[1]{One among you has been replaced with a doppelganger. Find it!}
		\tabpar[1]{\textbf{TN:} MAG 6 + [TIER] \tripip}\skipline
		
		\textbf{14. Zombies}\\
		
		\tabpar[1]{Living dead remains of the old villagers who didn’t leave.}
		\tabpar[1]{\textbf{TN:} ATK 6 + [TIER] \tripip}\skipline
		
		\textbf{15. Broken Wagon}\\
		
		\tabpar[1]{Someone must have beaten its horses.}
		\tabpar[1]{\textbf{TN:} DEF 6 + [TIER] \tripip}\skipline
		
		\columnbreak
		
		\textbf{16. Booby Trap}\\
		
		\tabpar[1]{Seems pretty hard to miss in hindsight, but you fell on it.}
		\tabpar[1]{\textbf{TN:} SPD 6 + [TIER] \cirpip}\skipline
		
		\textbf{17. Plague Pit}\\
		
		\tabpar[1]{The resting place of the ruin’s original inhabitants.}
		\tabpar[1]{\textbf{TN:} DEF 7 + [TIER] \cirpip}\skipline
		
		\textbf{18. Derelict Church}\\
		
		\tabpar[1]{An abandoned church. Something feels... wrong.}
		\tabpar[1]{\textbf{TN:} MAG 8 + [TIER] \tripip}
	
\end{multicols}

\textbf{BOSS: Ancient Robot} -- A machine made by the original villagers. Try to deactivate it.\\
\textbf{TN:} MAG 11 \cirpip\& DEF 12 \tripip

\pagebreak
	
	\chapter{Demon Lord's Castle}

\thispagestyle{empty}

All of your efforts, all of your sacrifices, have led you here. At the end of the world, the Demon Lord’s Castle awaits. One day it will rise up from the deep to swallow the world in darkness.

\skipline

\textbf{Unless you stop it.}

\skipline

The labyrinthine corridors within are the last challenge you must win before coming face to face with the Demon Lord. Leave Rosethorn Keep behind and take the march alongside your companions to finish your quest and save the world

\skipline

The presence of the Demon Lord has been plaguing the land around Rosethorn Keep and beyond for a very long time – but he has only grown in power since.

\skipline

After reaching Tier 5 in your Great Hall, you unlock the ability to vanquish the evil of the Demon Lord. Bring all of your companions. There are seven events to complete, and each success will reduce the TNs of both Demon Lord checks by one.

\skipline %
\skipline %

\begin{adjustbox}{center}
	\textbf{\large{I believe in you, hero!}}
\end{adjustbox}

\pagebreak

\subsection{Defeat the Demon Lord}

\begin{multicols}{2}
		
		\textbf{3. Imp}\\
		
		\tabpar[1]{A small and weak demon. It might let you pass if you scare it.}
		\tabpar[1]{\textbf{TN:} CHA 11 \cirpip}\skipline
		
		\textbf{4. Trophy Room}\\
		
		\tabpar[1]{Filled with horrifying trophies and conquests of the Demon Lord.}
		\tabpar[1]{\textbf{TN:} DEF 11 \cirpip}\skipline
		
		\textbf{5. Trapped Hall}\\
		
		\tabpar[1]{You are stuck in a corridor with dozens upon dozens of traps.}
		\tabpar[1]{\textbf{TN:} SPD 12 \cirpip}\skipline
		
		\textbf{6. Cultists}\\
		
		\tabpar[1]{They believe they will be spared if they worship the Demon Lord.}
		\tabpar[1]{\textbf{TN:} MAG 12] \tripip}\skipline
		
		\textbf{7. Miasma}\\
		
		\tabpar[1]{The mist of sickness and death inundates the halls of the castle.}
		\tabpar[1]{\textbf{TN:} DEF 13 \cirpip}\skipline
		
		\columnbreak
		
		\textbf{8. Prisoners}\\
		
		\tabpar[1]{Set them free, but assure that you will keep them safe.}
		\tabpar[1]{\textbf{TN:} CHA 11 \cirpip}\skipline
		
		\textbf{9. Devil Butler}\\
		
		\tabpar[1]{A servant of the Demon Lord who will die protecting his master.}
		\tabpar[1]{\textbf{TN:} ATK 12 \tripip}\skipline
		
		\textbf{10. Hellfire}\\
		
		\tabpar[1]{Unholy flames surround you, trying to scorch your flesh and soul.}
		\tabpar[1]{\textbf{TN:} DEF 13 \cirpip}\

		
		\textbf{11. Hallway of Living? Armor}\\
		
		\tabpar[1]{A long hallway lined with suits of armor. Did that one move?}
		\tabpar[1]{\textbf{TN:} SPD 12 \tripip}\skipline
		
		\textbf{12. Basilisk}\\
		
		\tabpar[1]{Beware its petrifying breath.}
		\tabpar[1]{\textbf{TN:} DEF 14 \tripip}\skipline
		
		\columnbreak
		
		\textbf{13. Alchemy Lab}\\
		
		\tabpar[1]{Reagents and materials fill the cabinets. Some may be helpful...}
		\tabpar[1]{\textbf{TN:} MAG 12 \cirpip}\skipline
		
		\textbf{14. Chimera}\\
		
		\tabpar[1]{An artificial amalgamation of creatures, full of agony and wrath.}
		\tabpar[1]{\textbf{TN:} ATK 13 \tripip}\skipline
		
		\textbf{15. Lost Knight}\\
		
		\tabpar[1]{Crestfallen and unwilling to proceed, he may be able to assist.}
		\tabpar[1]{\textbf{TN:} CHA 11 \cirpip}\skipline
		
		\columnbreak
		
		\textbf{16. Grand Library}\\
		
		\tabpar[1]{Filled with ancient and lost knowledge. Don’t delve too deep.}
		\tabpar[1]{\textbf{TN:}MAG 12 \cirpip}\skipline
		
		\textbf{17. Demon Warrior}\\
		
		\tabpar[1]{A soldier from hell and a slayer of countless men.}
		\tabpar[1]{\textbf{TN:} ATK 14 \tripip}\skipline
		
		\textbf{18. Endless Stairs}\\
		
		\tabpar[1]{Rumors say if you jump back quickly enough, you will be able to escape.}
		\tabpar[1]{\textbf{TN:} SPD 13 \cirpip}
		
\end{multicols}

\textbf{BOSS: DEMON LORD} -- Defeat the Demon Lord and banish him -- or the world will be his.\\
\textbf{TN:} ATK 20 \tripip \& MAG 20 \cirpip

\skipline

If you defeat the demon lord, turn the page. If not...

\skipline
\skipline

\begin{adjustbox}{center}
	\textbf{Try Again.}
\end{adjustbox}

\pagebreak

\thispagestyle{empty}
\topskip0pt
\vspace*{\fill}
\begin{adjustbox}{center}
	\BVerbatimInput{img/coins.txt}
\end{adjustbox}
\begin{adjustbox}{center}
	\textbf{CONGRATULATIONS!}
\end{adjustbox}
\begin{adjustbox}{center}
	You have defeated the \textbf{DEMON LORD}!
\end{adjustbox}
\begin{adjustbox}{center}
	You gain: A Saved World (and so much gold).
\end{adjustbox}
\vspace*{\fill}
%

	
	\pagebreak
	
	\uheader{Map of Rosethorn Keep}
	
	\pagestyle{empty}
	
	\begin{adjustbox}{fbox=1pt}
		\begin{tikzpicture}
			\draw[step=1cm,color=black,dashed] (-1,-1) grid (10,5);
		\end{tikzpicture}
	\end{adjustbox}
	
	\uheader{Companions}
	\begin{retrotable}[|l|L|l|c|c|c|c|c|c|][\tablex\linewidth]
		\multicolumn{9}{_l_}{\linecharw[\_]{\tablex\linewidth}}\\
		\# & Companion & Stars & ATK & DEF & MAG  & CHA & SPD & Party \\ \hline
		1  & Archer        & *  & 2 & 1 & 2 & 2 & 3& ~   \\\hline
		2  & Arriaga the Rogue         & * & 3 & 2 & 1 & 2 & 2 & ~ \\\hline
		3  & Mathias the Squire      & *  & 2 & 3 & 2 & 2 & 1 & ~\\\hline
		4  & Chime the Bard          & *  & 1 & 2 & 2 & 3 & 2 & ~  \\\hline
		5  & Ellen the Warrior          & *  & 2 & 3 & 1 & 2 & 2 & ~  \\\hline
		6  & Lichen the Sling     & * & 3 & 1 & 2 & 2 & 2 & ~  \\\hline
		7  & Cassandra the Witch       & **  & 1 & 1 & 3 & 3 & 2& ~ \\\hline
		8  & Diego the Duelist       & **  & 3 & 1 & 1 & 2 & 3 & ~ \\\hline
		9  & Astrid, Shield Maiden  & ** & 3 & 3 & 1 & 1 & 2& ~ \\\hline
		10 & Eryn the Nun        & **     & 1 & 2 & 3 & 3 & 1 & ~ \\\hline
		11 & Scarlet, Plague Doc  & *** & 1 & 3 & 4 & 1 & 1& ~ \\\hline
		12 & Gottfried the Knight     & ***      & 3 & 4 & 1 & 1 & 1  & ~\\ \hline
		~ &~ &~ & ~& ~ & ~& ~& ~& ~\\ \hline
		~ &~ &~ & ~& ~ & ~& ~& ~& ~\\ \hline
		~ &~ &~ & ~& ~ & ~& ~& ~& ~\\ \hline
		~ &~ &~ & ~& ~ & ~& ~& ~& ~\\ \hline
		~ &~ &~ & ~& ~ & ~& ~& ~& ~\\ 
		\multicolumn{9}{^l^}{\linecharw[\odash]{\tablex\linewidth}}\\
	\end{retrotable}
	
	\begin{figure}[H]%
		\uheader{Resources}
		\subfloat{{	\begin{retrotable}[|L|l|][\tablesm\linewidth]
	\multicolumn{2}{_l_}{\linecharw[\_]{\tablesm\linewidth}}\\
	Resource & Amount  \\ \hline
	Gold &~ \\ \hline
	Iron &~ \\ \hline
	Leather &~ \\ \hline
	Lumber &~ \\ \hline
	Stone &~ \\ \hline
	~ &~ \\ \hline
	~ &~ \\ \hline
	~ &~ \\ \hline
	~ &~ \\ \hline
	~ &~ \\ \hline
	~ &~ \\ \hline
	~ &~ \\ \hline
	~ &~ \\ \hline
	~ &~ \\ \hline
	~ &~ \\ \hline
	~ &~ \\ \hline
	~ &~ \\ \hline
	~ &~ \\ \hline
	~ &~ \\ 
	\multicolumn{2}{^l^}{\linecharw[\odash]{\tablesm\linewidth}}\\
\end{retrotable} }}%
		\subfloat{{	\begin{retrotable}[|l|L|l|][\tablesm\linewidth]
	\multicolumn{3}{_l_}{\linecharw[\_]{\tablesm\linewidth}}\\
	Room & Resources & Rank  \\ \hline
	Great Hall & G \& S  &~ \\ \hline
	Armory & I \& Lu  &~ \\ \hline
	Bedrooms & Le \& G &~ \\ \hline
	Courtyard & S \& I &~ \\ \hline
	Kitchen & S \& Lu &~ \\ \hline
	Stables & Le \& Lu & ~ \\ \hline
	~ &~&~ \\ \hline
	~ &~&~ \\ \hline
	~ &~&~ \\ \hline
	~ &~&~ \\ \hline
	~ &~&~ \\ \hline
	~ &~&~ \\ \hline
	~ &~&~ \\ \hline
	~ &~&~ \\ \hline
	~ &~&~ \\ \hline
	~ &~&~ \\ \hline
	~ &~&~ \\ \hline
	~ &~&~ \\ \hline
	~ &~&~ \\ 
	\multicolumn{3}{^l^}{\linecharw[\odash]{\tablesm\linewidth}}\\
\end{retrotable} }}%
	\end{figure}
	
	\notes{10}{\textwidth}
	
	\notes{36}{\textwidth}
	
	\begin{tikzpicture}
		\draw[step=1cm,color=black,dashed] (-1,-1) grid (10,15);
	\end{tikzpicture}
	
		\begin{tikzpicture}
		\draw[step=1cm,color=black,dashed] (-1,-1) grid (10,15);
	\end{tikzpicture}
	

	
\end{document}