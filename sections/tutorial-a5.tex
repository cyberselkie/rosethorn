\section{Tutorial}

\tabpar[5]{\textbf{Objective:} Clear Rosethorn Keep}

Nestled deep in the forest, amongst ancient crypts and haunted woods, lies the legendary Rosethorn Keep. Largely in ruins, like any remains of civilization that was once around it, a significant amount of work is required to turn it into a flourishing castle once again. Despite its remote location, it will be a key stronghold in the fight against the Demon Lord.

\skipline

You have been tasked with recruiting adventurers, clearing the keep, and restoring it to its former glory. Do you have what it takes?

\skipline

Your first task, brave hero, is to recruit three companions to join you on your quest. These adventurers will clear out the dungeons, crypts, and other nefarious locations that surround the castle, and return with the Resources you need to rebuild it.

\skipline

\begin{adjustbox}{center}
	\textbf{Roll 1d6 three times, rerolling any duplicates.}
\end{adjustbox}

\skipline

Match the number on the die with the list of adventurers. These are the three companions you have successfully recruited.

\pagebreak

Their stat blocks are following. Notice that there are adventurers that you can’t use yet. This is as intended, and will be explained later.

\subsection{Companions}
Each companion has a \textbf{Trait}, a skill they can use once in every dungeon, and a number assigned to their \textbf{Stats}. If you go into a dungeon with two characters of the same Trait, any character can use that trait. It may only be used once per character with the trait.

\skipline

\begin{adjustbox}{center}
	\begin{retrotable}[~c|L|c~c~c~c~c|l~][1.0\linewidth]
		\# & Adventurer & ATK & DEF & MAG  & CHA & SPD & Stars\\ \Xhline{2\arrayrulewidth}
		1  & Archer          & 2 & 1 & 2 & 2 & 3 & *  \\
		\multicolumn{8}{l}{\linechars[-]{3} \emph{Trait:} Use SPD instead of DEF.}\\ \Xhline{2\arrayrulewidth}
		2  & Arriaga the Rogue          & 3 & 2 & 1 & 2 & 2 & *  \\
		\multicolumn{8}{l}{\linechars[-]{3} \emph{Trait:} Use ATK instead of MAG.}\\ \Xhline{2\arrayrulewidth}
		3  & Mathias the Squire        & 2 & 3 & 2 & 2 & 1 & * \\
		\multicolumn{8}{l}{\linechars[-]{3} \emph{Trait:} -1 from TN of next event.}\\ \Xhline{2\arrayrulewidth}
		4  & Chime the Bard            & 1 & 2 & 2 & 3 & 2 & * \\
		\multicolumn{8}{l}{\linechars[-]{3} \emph{Trait:} +1 to any stat.}\\ \Xhline{2\arrayrulewidth}
		5  & Ellen the Warrior           & 2 & 3 & 1 & 2 & 2 & * \\
		\multicolumn{8}{l}{\linechars[-]{3} \emph{Trait:} Use DEF instead of SPD.}\\ \Xhline{2\arrayrulewidth}
		6  & Lichen the Sling       & 3 & 1 & 2 & 2 & 2 & * \\
		\multicolumn{8}{l}{\linechars[-]{3} \emph{Trait:} +1 to any stat.}\\ \Xhline{2\arrayrulewidth}
		7  & Cassandra the Witch         & 1 & 1 & 3 & 3 & 2 & **\\
		\multicolumn{8}{l}{\linechars[-]{3} \emph{Trait:} Use MAG instead of ATK.}\\ \Xhline{2\arrayrulewidth}
		8  & Diego the Duelist         & 3 & 1 & 1 & 2 & 3 & ** \\
		\multicolumn{8}{l}{\linechars[-]{3} \emph{Trait:} Use SPD instead of DEF.}\\ \Xhline{2\arrayrulewidth}
		9  & Astrid, Shield Maiden   & 3 & 3 & 1 & 1 & 2 & **\\
		\multicolumn{8}{l}{\linechars[-]{3} \emph{Trait:} Use DEF instead of SPD.}\\ \Xhline{2\arrayrulewidth}
		10 & Eryn the Nun             & 1 & 2 & 3 & 3 & 1 & **\\
		\multicolumn{8}{l}{\linechars[-]{3} \emph{Trait:} -1 from TN of next event.}\\ \Xhline{2\arrayrulewidth}
		11 & Scarlet, Plague Doctor  & 1 & 3 & 4 & 1 & 1 & *** \\
		\multicolumn{8}{l}{\linechars[-]{3} \emph{Trait:} Use MAG instead of ATK.}\\ \Xhline{2\arrayrulewidth}
		12 & Gottfried the Knight           & 3 & 4 & 1 & 1 & 1 & *** \\
		\multicolumn{8}{l}{\linechars[-]{3} \emph{Trait:} Use ATK instead of MAG.}\\
	\end{retrotable}
\end{adjustbox}

\skipline



Now that you have recruited your companions and arrived at the Keep, a challenge glares back at you: This castle has been abandoned for many years. It must be fully secured and cleared of nefarious things before you can start anything else in your mission.

\begin{adjustbox}{center}
	\textbf{Roll 1d6 three times, and record each result.}
\end{adjustbox}
\begin{adjustbox}{center}
	\textbf{You will use these later.}
\end{adjustbox}

\skipline

Once you have rolled your dice, you start creeping through the halls and rooms of the Keep. Soon, you come across your first event...

\begin{adjustbox}{center}
	\textbf{Roll 1d6, and compare the result to the list of events.}
\end{adjustbox}

\subsection[Clearing the Keep]{Clearing the Keep}
\begin{tabbox}[3][3]
	
	\textbf{1. Stuck Door}\\
	
	\tabpar[1]{Door stuck! Please, I beg you, kick it down!}
	\tabpar[1]{\textbf{TN:} ATK 4 \cirpip}\skipline
	
	\textbf{2. Derelict Armory}\\
	
	\tabpar[1]{Old weapons hang precariously from the wall.}
	\tabpar[1]{\textbf{TN:} DEF 4 \cirpip}\skipline
	
	\textbf{3. Haunted Painting}\\
	
	\tabpar[1]{Feels like if you jumped on it, you would go somewhere far away...}
	\tabpar[1]{\textbf{TN:} MAG 5 \cirpip}\skipline
	
	\textbf{4. Castle Ghost}\\
	
	\tabpar[1]{Doesn’t seem aggressive, maybe you can ask it for directions.}
	\tabpar[1]{\textbf{TN:} CHA 5 \cirpip}\skipline
	
	\textbf{5. Doorway Trap}\\
	
	\tabpar[1]{A brick placed on top of an ajar door.}
	\tabpar[1]{\textbf{TN:} SPD 5 \cirpip}\skipline
	
	\textbf{6. Living Armor}\\
	
	\tabpar[1]{A living suit of armor, guarding the keep from intruders.}
	\tabpar[1]{\textbf{TN:} ATK 6 \tripip}\skipline
\end{tabbox}
\skipline

Now you must complete each event, for better or worse – take any of the numbers you rolled and recorded previously, and choose a character with a suitable stat for the event. Add the recorded die result and the companion’s stat together. If the total meets or beats the TN, you have successfully completed the event! If you aren’t able to succeed, check the \textbf{Traits} of your companions.

\skipline

Using traits is a great way to succeed at an event you might not have been able to otherwise. Each companion can be used twice in a dungeon, so you may run through some without using some of them at all! However, if a companion is defeated, they may not be used again, and must rest in the castle before they can go out once more.

\skipline

After you have completed your first event, roll again, and complete the second one. If you roll an event that is duplicate to one you have rolled before, you automatically succeed, but in later dungeons, this will not count towards your resource gain. You complete as many events, one at a time, as companions you have brought with you. For the first dungeon this is three.

\skipline
Even if you don’t succeed in every event, once you have completed all three, the Keep is yours.

\skipline

\begin{adjustbox}{center}
	\textbf{CONGRATULATIONS!}
\end{adjustbox}
\begin{adjustbox}{center}
	You have earned: \emph{Rosethorn Keep}
\end{adjustbox}
\begin{adjustbox}{center}
	Your \textbf{Great Hall} has been rebuilt to Tier 1.
\end{adjustbox}

\skipline

Wait... Your Great Hall? What’s that?

\skipline

Your castle has six sections, each providing a different benefit for your companions, and requiring a different amount of Resources to build and upgrade. When you cleared the castle, you found that the Great Hall was still standing, and in decent condition. (More about the castle is found on the next page.)

\skipline

The rest of the castle is in shambles, and will need Resources to restore it to its former glory. 
To collect Resources, you delve into dungeons and overcome the events within. The type of dungeon you choose determines what Resources you get, and the Resources you stockpile in turn determine what parts of the castle you can rebuild and upgrade. Each event in a dungeon is marked by a \tripip  (triangle) or a \cirpip (circle). The \tripip marks a combat encounter, and the \cirpip marks a non-combat encounter.